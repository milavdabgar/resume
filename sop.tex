%-------------------------
% Statement of Purpose
% Author: Milav Jayeshkumar Dabgar
% AICTE Industry Fellowship Application - 2025
% Compiled with XeLaTeX for best results
%-------------------------

\documentclass[12pt,a4paper]{article}

% Packages for XeLaTeX
\usepackage{fontspec}
\usepackage{xunicode}
\usepackage{xltxtra}

% Modern fonts
\setmainfont{Liberation Serif}[Scale=1.0]
\setsansfont{Liberation Sans}[Scale=1.0]
\setmonofont{Liberation Mono}[Scale=1.0]

% Packages
\usepackage[top=1in, bottom=1in, left=1in, right=1in]{geometry}
\usepackage{xcolor}
\usepackage{titlesec}
\usepackage{fontawesome5}
\usepackage{hyperref}
\usepackage{setspace}
\usepackage{parskip}

% Colors
\definecolor{primary}{RGB}{0, 79, 144}
\definecolor{secondary}{RGB}{45, 45, 45}
\definecolor{accent}{RGB}{0, 102, 204}

% Hyperref setup
\hypersetup{
    colorlinks=true,
    linkcolor=primary,
    urlcolor=primary,
    pdfauthor={Milav Jayeshkumar Dabgar},
    pdftitle={Statement of Purpose - AICTE Industry Fellowship - Milav Dabgar},
    pdfsubject={AICTE Industry Fellowship Application},
    pdfkeywords={AICTE Fellowship, Industry Fellowship, AI, Data Science, Engineering Education, Innovation, R&D}
}

% Line spacing
\onehalfspacing

% Custom section formatting
\titleformat{\section}
    {\color{primary}\Large\sffamily\bfseries}
    {}
    {0em}
    {}[{\color{primary}\titlerule[1pt]\vspace{2pt}}]

% Header design
\newcommand{\makeheader}{
    \begin{center}
        {\Huge\sffamily\bfseries\color{primary} Statement of Purpose}
        
        \vspace{10pt}
        {\Large\sffamily\color{secondary} AICTE Industry Fellowship Application -- 2025}
        
        \vspace{15pt}
        {\large\sffamily\bfseries\color{accent} Milav Jayeshkumar Dabgar}
        
        \vspace{8pt}
        {\normalsize\color{secondary} Engineering Educator \& R\&D Professional}
        
        \vspace{8pt}
        {\small\color{secondary}
        \faIcon{envelope}\,\href{mailto:milav.dabgar@gmail.com}{\color{primary}milav.dabgar@gmail.com} $\bullet$
        \faIcon{phone}\,+91 8128576285 $\bullet$
        AICTE ID: 1-3241967546}
        
        \vspace{5pt}
        \rule{0.8\textwidth}{0.5pt}
    \end{center}
    \vspace{15pt}
}

%-------------------------------------------
%%%%%%  DOCUMENT STARTS HERE  %%%%%%%%%%%%%%%%%%%%%%%%%%%%
%-------------------------------------------

\begin{document}

\makeheader

My engineering journey began with a deep-rooted curiosity about how things work—a habit of dismantling toys as a child transformed into a lifelong pursuit of building and optimizing real systems. This hands-on inclination led me to pursue a degree in Electronics and Communication Engineering and kick-start my career in the R\&D division of a Japan-based firm, where I managed end-to-end electronics development for commercial embedded products. Working directly on circuit simulation, PCB design, firmware development, and control systems gave me a strong foundation not only in hardware but in the systems-thinking approach that connects devices to intelligent behavior.

In 2016, I transitioned into academics and have since been serving as a Lecturer at Government Polytechnic, Palanpur. Over these years, my technical interests evolved beyond embedded systems toward software engineering, artificial intelligence, and data-driven automation. I complemented this transition by completing over 100 online courses and enrolling in the \textbf{BS in Data Science and Applications} from \textbf{IIT Madras}, where I've earned the \textbf{Diploma in Programming} and am one project away from completing the \textbf{Diploma in Data Science}. This experience gave structure and depth to my evolving interest in AI, web development, and applied machine learning.

In my teaching role, I engage with students not just in classrooms but also through hands-on innovation. I have mentored several project teams working in IoT, drones, and embedded automation. One of these efforts resulted in two student patents—one of which received over ₹25 lakhs in funding from Shark Tank India and another ₹20 lakhs through government innovation support. Beyond teaching, I contribute to institutional development as the \textbf{IT Convener}, \textbf{SSIP Co-Convener}, \textbf{Training \& Placement Member}, and part of our \textbf{MIS and UDAYAM initiative teams}. These responsibilities reflect my ability to coordinate across departments, lead technical initiatives, and foster a culture of innovation.

To address gaps in academic systems, I initiated and now lead the development of a \textbf{Next.js-based smart academic portal}—an in-house solution offering smart attendance tracking, assessment workflows, committee and profile management, automated result processing, and real-time feedback analytics. The system is developed collaboratively with student contributions through Git, and is deployed using a self-hosted Linux server stack with Dockerized services, CI/CD pipelines, and backup infrastructure—all managed by myself. This project reflects not only my full-stack and infrastructure expertise, but also my dedication to real-world, scalable solutions that blend software, operations, and educational workflows.

These cumulative experiences—spanning electronics, data systems, AI/ML practices, and full-stack engineering—have shaped me into a hands-on, self-motivated, and systems-oriented engineer-educator. I believe the Industry Fellowship offers a transformative platform to further this trajectory by immersing myself in applied industrial research, working on real-world challenges in areas such as edge intelligence, backend optimization, or AI system integration. The opportunity to collaborate with industry professionals and researchers will expose me to high-performance team dynamics, product-oriented development cycles, and problem-solving frameworks beyond the academic context.

Long-term, I aspire to pursue a Ph.D. in the fields of AI, embedded computing, or system design from a premier institution. I see this fellowship as a bridge to that goal—a chance to bring industrial rigor into my academic evolution. More importantly, I aim to channel this experience back into the diploma-level ecosystem in Gujarat, which, in my view, holds vast untapped potential for innovation and research-driven education. As a faculty leader already involved in SSIP and IT strategy, I'm committed to shaping student initiatives, research culture, and interdisciplinary collaboration.

I am confident that my technical depth, academic versatility, leadership experience, and motivation for continuous learning align well with the goals of the Industry Fellowship. I look forward to contributing meaningfully while growing professionally and academically through this opportunity.

\vspace{20pt}

\begin{flushright}
\begin{minipage}{0.4\textwidth}
    \textbf{Milav Jayeshkumar Dabgar} \\
    Lecturer, EC Department \\
    Government Polytechnic, Palanpur \\
    AICTE Faculty ID: 1-3241967546 \\
    \href{mailto:milav.dabgar@gmail.com}{\color{primary}milav.dabgar@gmail.com}
\end{minipage}
\end{flushright}

\end{document}